%-------------------------
% Resume in LaTeX
%
% Author       : Mikel Pintado
% Based off of : https://github.com/sb2nov/resume & https://github.com/jakegut/resume
% License      : MIT
%------------------------

\documentclass[letterpaper,11pt]{article}

\usepackage{latexsym}
\usepackage[empty]{fullpage}
\usepackage{titlesec}
\usepackage{marvosym}
\usepackage[usenames,dvipsnames]{color}
\usepackage{verbatim}
\usepackage{enumitem}
\usepackage[hidelinks]{hyperref}
\usepackage{fancyhdr}
\usepackage{tabularx}
\usepackage{expl3}
\usepackage[american,spanish]{babel}

\providecommand{\langver}{en}
\newcommand{\iflangver}[2]{\ifdefstring{\langver}{#1}{#2}{}}

\pagestyle{fancy}
\fancyhf{}
\fancyfoot{}
\renewcommand{\headrulewidth}{0pt}
\renewcommand{\footrulewidth}{0pt}

\addtolength{\oddsidemargin}{-0.5in}
\addtolength{\evensidemargin}{-0.5in}
\addtolength{\textwidth}{1in}
\addtolength{\topmargin}{-.5in}
\addtolength{\textheight}{1.0in}

\urlstyle{same}

\raggedbottom
\raggedright
\setlength{\tabcolsep}{0in}

\titleformat{\section}{
  \vspace{-4pt}\scshape\raggedright\large
}{}{0em}{}[\color{black}\titlerule \vspace{-5pt}]

\newcommand{\resumeItem}[1]{
  \item\small{
    {#1 \vspace{-2pt}}
  }
}

\newcommand{\resumeSubheading}[4]{
  \vspace{-2pt}\item
    \begin{tabular*}{0.97\textwidth}[t]{l@{\extracolsep{\fill}}r}
      \textbf{#1} & #2 \\
      \textit{\small#3} & \textit{\small #4} \\
    \end{tabular*}\vspace{-7pt}
}

\newcommand{\resumeSubSubheading}[2]{
    \item
    \begin{tabular*}{0.97\textwidth}{l@{\extracolsep{\fill}}r}
      \textit{\small#1} & \textit{\small #2} \\
    \end{tabular*}\vspace{-7pt}
}

\newcommand{\resumeProjectHeading}[2]{
    \item
    \begin{tabular*}{0.97\textwidth}{l@{\extracolsep{\fill}}r}
      \small#1 & #2 \\
    \end{tabular*}\vspace{-7pt}
}

\newcommand{\resumeSubItem}[1]{\resumeItem{#1}\vspace{-4pt}}

\renewcommand\labelitemii{$\vcenter{\hbox{\tiny$\bullet$}}$}

\newcommand{\resumeSubHeadingListStart}{\begin{itemize}[leftmargin=0.15in, label={}]}
\newcommand{\resumeSubHeadingListEnd}{\end{itemize}}
\newcommand{\resumeItemListStart}{\begin{itemize}}
\newcommand{\resumeItemListEnd}{\end{itemize}\vspace{-5pt}}

\begin{document}

\newcommand{\name}[1]{\textbf{\Huge \scshape #1}}
\ExplSyntaxOn
\NewDocumentCommand{\phone}{m}
{
	\tl_set:Nn\l_tmpa_tl{#1}
	\regex_replace_all:nnN
	{(\+\d+)?(\d{9})}
	{\1\ \2}
	\l_tmpa_tl
	\tl_use:N \l_tmpa_tl
}
\ExplSyntaxOff
\newcommand{\email}[1]{\href{mailto:#1}{\underline{#1}}}
\newcommand{\linkedin}[1]{\href{https://linkedin.com/in/#1}{\underline{linkedin.com/in/#1}}}
\newcommand{\github}[1]{\href{https://github.com/#1}{\underline{github.com/#1}}}
\newcommand{\site}[1]{\href{https://#1}{\underline{#1}}}

\begin{center}
	\name{Mikel Pintado del Campo}
	\\ \vspace{2pt}

	\small{
		\phone{+34651459885} $|$
		\email{mikelpint@protonmail.com} $|$
		\linkedin{mikelpint}
		\github{mikelpint}
		% \site{mikelpint.com}
	}
\end{center}

\iflangver{en}{\section{Education}}
\iflangver{es}{\section{Educación}}

\resumeSubHeadingListStart

\iflangver{en}
{
	\resumeSubheading
	{University of Deusto}{Sep. 2021 -- Sep. 2025}
	{Bachelor's Degree in Computer Engineering}{Bilbao, Spain}
}
\iflangver{es}
{
	\resumeSubheading
	{Universidad de Deusto}{Sep. 2021 -- Sep. 2025}
	{Grado en Ingeniería Informática}{Bilbao, España}
}

\resumeSubHeadingListEnd

\iflangver{en}{\section{Professional experience}}
\iflangver{es}{\section{Experiencia profesional}}

\resumeSubHeadingListStart

\iflangver{en}{
	\resumeSubheading
	{Undergraduate fullstack developer}{Mar. 2023 -- Jun. 2025}
	{Discovery Space (via DEUSTEK)}{Bilbao, Spain}
}
\iflangver{es}{
	\resumeSubheading
	{Desarrollador \textit{fullstack} bajo beca universitaria}{Mar. 2023 -- Jun. 2025}
	{Discovery Space (vía DEUSTEK)}{Bilbao, España}
}
\resumeItemListStart
\iflangver{en}{
	\resumeItem{
		Sole developer of an interactive scenario-based conversational agent-assisted e-learning platform intended for
		EU-wide usage (excluding its AI-based capabilities).
	}
	\resumeItem{
		Co-authored a 13-page paper titled "Learning platform with remote labs guided by an intelligent conversational
		agent" for the \textit{expat’25} (held on September 5, 2025).
	}
	\resumeItem{
		Developed a proprietary JSON-based specification for interactive scenarios that has been used and is still used
		throughout many derived projects.
	}
	\resumeItem{
		Close collaboration with other research institutions and companies across many European countries in order to
		coordinate the developments.
	}
	\resumeItem{
		Results showcased across many conferences and venues such as Balkan Summer Schools, conferences and consortium
		events.
	}
	\resumeItem{
		Developed a nuanced RESTful API and its respective web-based client using a varied stack of technologies like
		NestJS, MongoDB, Redis, TimescaleDB, HAProxy, Docker, Next.js and many more.
	}
}
\iflangver{es}{
	\resumeItem{
		Único desarrollador de una plataforma educativa digital con la intención de ser usada en la Unión Europea basada
		en escenarios interactivos con asistencia de agente conversacional (excluyendo las capacidades de inteligencia
		artificial).
	}
	\resumeItem{
		Coautor de un artículo de 13 páginas con título "Learning platform with remote labs guided by an intelligent
		conversational agent" para ser expuesto en la conferencia \textit{expat'25} (tuvo lugar el 5 de septiembre de
		2025).
	}
	\resumeItem{
		Desarrollo de un estándar propietario basado en JSON para la especificación de escenarios interactivos que ha
		sido y sigue siendo usado por proyectos derivados.
	}
	\resumeItem{
		Desarrollo de una compleja y detallada API 'RESTful' y su respectivo cliente basado en web usando una variada
		pila de tecnologías como NestJS, MongoDB, Redis, TimescaleDB, HAProxy, Docker, Next.js y muchas más.
	}
}
\resumeItemListEnd

\resumeSubHeadingListEnd

\iflangver{en}{\newcommand{\asm}[1]{#1 assembly}}
\iflangver{es}{\newcommand{\asm}[1]{ensamblador de #1}}

\newcommand{\pl}{
	C, C++, Java, JavaScript/TypeScript, Rust, HTML, (S)CSS, SQL, Nix, Python, R, \asm{RISC-V}, \asm{x86\_64},
	Unix shell \iflangver{en}{and}\iflangver{es}{y} LaTeX.
}
\newcommand{\lib}{
	Git, UNIX/Linux, NixOS, \textit{systemd}, Docker, Vagrant, QEMU, Ansible, SQLite, PostgreSQL (TigerData), MongoDB,
	Redis, Prisma, Apache, Nginx, HAProxy, GMP/MPFR/MPC, Make, Meson, Maven, Gradle, Apache Ant, Preact/React, Next.js,
	Tailwind CSS, Bootstrap, Sass, Node.js, Deno (Fresh), Express, NestJS, GCC, Clang, TCC, MSVC, Valgrind, GDB, cURL,
	\textit{nftables}, Turborepo, JUnit, Jest, Spring Boot, Keepalived, SSH, \textit{libnm}, \textit{ncurses}, ...
}

\iflangver{en}{\section{Technical Skills}}
\iflangver{es}{\section{Habilidades Técnicas}}

\begin{itemize}[leftmargin=0.15in, label={}]
	\small{\item{
	      \iflangver{en}{\textbf{Computer languages}{: \pl}}
	      \iflangver{es}{\textbf{Lenguajes informáticos}{: \pl}}
	      \\

	      \iflangver{en}{\textbf{Frameworks, libraries and technologies}{: \lib}}
	      \iflangver{es}{\textbf{Frameworks, bibliotecas y tecnologías}{: \lib}}
	      \\

	      \iflangver{en}{\textbf{Languages}{: Spanish (native), English (Cambridge C1), Basque (B2 equivalent)}}
	      \iflangver{es}{\textbf{Idiomas}{: castellano (nativo), inglés (Cambridge C1), vasco (equivalente a B2)}}
	      }}
\end{itemize}

\iflangver{en}{
	\section{Professional and academic interests}

	 {
	  Embedded systems, system administration, (mostly POSIX-compliant) operating systems, virtualization, compilers,
	  (programming) languages, linear algebra, scientific computing, numerical methods, cryptography, algorithms,
	  declarative systems, parallelization/vectorization, distributed computing, real-time computing, data structures,
	  optimization and caching, computer networks, type systems, validation and formal verification,
	  theorethical computer science, retro/vintage computing, software ethics and licenses, ...
	 }
}
\iflangver{es}{
	\section{Intereses profesionales y académicos}

	 {
	  Sistemas embebidos, administración de sistemas, sistemas operativos, virtualización, compiladores,
	  lenguages (de programación), álgebra lineal, computación científica, métodos numéricos, criptografía, algoritmos,
	  sistemas declarativos, paralelización/vectorization, computación distribuida, computación en tiempo real,
	  estructuras de datos, optimización y cacheo, redes de computadores, sistemas de tipos,
	  validación y verificación formal, computación teórica, informática antigua/retro, ética del software y licencias,
	  ...
	 }
}

\end{document}
